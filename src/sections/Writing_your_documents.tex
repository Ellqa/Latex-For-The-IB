% Copyright 2018-2020 ImmortalPharaoh7, Bryce AS202313
%
% This file is part of Latex-For-The-IB.
%
% Latex-For-The-IB is free software: you can redistribute it and/or modify it
% under the terms of the GNU General Public License as published by
% the Free Software Foundation, either version 3 of the License, or
% (at your option) any later version.
%
% Latex-For-The-IB is distributed in the hope that it will be useful, but
% WITHOUT ANY WARRANTY; without even the implied warranty of
% MERCHANTABILITY or FITNESS FOR A PARTICULAR PURPOSE. See the GNU
% General Public License for more details.
%
% You should have received a copy of the GNU General Public License
% along with Latex-For-The-IB. If not, see http://www.gnu.org/licenses/.
%
\section{Writing your documents}
In this section we will focus on the essential tables / graphs you'll need when writing an IA.

\subsection{Tables}
\href{https://www.overleaf.com/learn/latex/Tables}{Overleaf's documentation on tables}
already provides almost everything you need to write a table.
\href{https://www.tablesgenerator.com/}{This website} generates \LaTeX{} tables
if you would like some sort of GUI assistance when dealing with tables.
But this guide is to offer boilerplate, so here's one for raw data tables
(note: this boilerplate requires the inclusion of the packages mentioned in packages section):
\begin{verbatim}
\begin{table}[H]
\begin{tabu} to \textwidth {X[c]X[c]X[c]X[c]X[c]X[c]X[c]X[c]}
\hline
\multirow{2}{*}{\parbox{2cm}{IV}}&
\multicolumn{6}{c}{Table Title} \\
\cline{2-8}
& Trial 1 & Trial 2 & Trial 3 & Trial 4 & Trial 5 & Average & Uncert \\
\hline
value & value & value & value & value & value & value & value \\
 & & & & & & & \\
 & & & & & & & \\
 & & & & & & & \\
 & & & & & & & \\
\hline
\end{tabu}
\caption{Table caption}
\end{table}
\end{verbatim}
Resulting table:
\begin{table}[H]
\begin{tabu} to \textwidth {X[c]X[c]X[c]X[c]X[c]X[c]X[c]X[c]}
\hline
\multirow{2}{*}{\parbox{2cm}{IV}}&
\multicolumn{6}{c}{Table Title} \\
\cline{2-8}
& Trial 1 & Trial 2 & Trial 3 & Trial 4 & Trial 5 & Average & Uncert \\
\hline
value & value & value & value & value & value & value & value \\
 & & & & & & & \\
 & & & & & & & \\
 & & & & & & & \\
 & & & & & & & \\
\hline
\end{tabu}
\caption{Table caption}
\end{table}

Just note that tables are highly customizable and we highly recommend
that you read Overleaf's guide as it will allow you to better understand the code we've used.

\subsection{Graphs}

\subsection{BibLaTeX}